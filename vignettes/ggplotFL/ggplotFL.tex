% tufte-handout.tex - DESC
% Iago Mosqueira - JRC. 2013
%
\documentclass{tufte-handout}

% ams
\usepackage{amssymb,amsmath}

% Set up the images/graphics package
\usepackage{graphicx}
\setkeys{Gin}{width=\linewidth,totalheight=\textheight,keepaspectratio}
\graphicspath{{graphics/}}

% natbib
\usepackage{natbib}
\bibliographystyle{plainnat}

% biblatex

% booktabs
\usepackage{booktabs}

% url
\usepackage{url}

% units.
\usepackage{units}

% fancyvrb
\usepackage{fancyvrb}
\fvset{fontsize=\normalsize}
\DefineShortVerb[commandchars=\\\{\}]{\|}
\DefineVerbatimEnvironment{Highlighting}{Verbatim}{commandchars=\\\{\}}


% multiplecol
\usepackage{multicol}

% lipsum
\usepackage{lipsum}

% These commands are used to pretty-print LaTeX commands
\newcommand{\doccmd}[1]{\texttt{\textbackslash#1}}% command name -- adds backslash automatically
\newcommand{\docopt}[1]{\ensuremath{\langle}\textrm{\textit{#1}}\ensuremath{\rangle}}% optional command argument
\newcommand{\docarg}[1]{\textrm{\textit{#1}}}% (required) command argument
\newenvironment{docspec}{\begin{quote}\noindent}{\end{quote}}% command specification environment
\newcommand{\docenv}[1]{\textsf{#1}}% environment name
\newcommand{\docpkg}[1]{\texttt{#1}}% package name
\newcommand{\doccls}[1]{\texttt{#1}}% document class name
\newcommand{\docclsopt}[1]{\texttt{#1}}% document class option name

% Shaded
\newenvironment{Shaded}{}{}
\newcommand{\KeywordTok}[1]{\textcolor[rgb]{0.00,0.44,0.13}{\textbf{{#1}}}}
\newcommand{\DataTypeTok}[1]{\textcolor[rgb]{0.56,0.13,0.00}{{#1}}}
\newcommand{\DecValTok}[1]{\textcolor[rgb]{0.25,0.63,0.44}{{#1}}}
\newcommand{\BaseNTok}[1]{\textcolor[rgb]{0.25,0.63,0.44}{{#1}}}
\newcommand{\FloatTok}[1]{\textcolor[rgb]{0.25,0.63,0.44}{{#1}}}
\newcommand{\CharTok}[1]{\textcolor[rgb]{0.25,0.44,0.63}{{#1}}}
\newcommand{\StringTok}[1]{\textcolor[rgb]{0.25,0.44,0.63}{{#1}}}
\newcommand{\CommentTok}[1]{\textcolor[rgb]{0.38,0.63,0.69}{\textit{{#1}}}}
\newcommand{\OtherTok}[1]{\textcolor[rgb]{0.00,0.44,0.13}{{#1}}}
\newcommand{\AlertTok}[1]{\textcolor[rgb]{1.00,0.00,0.00}{\textbf{{#1}}}}
\newcommand{\FunctionTok}[1]{\textcolor[rgb]{0.02,0.16,0.49}{{#1}}}
\newcommand{\RegionMarkerTok}[1]{{#1}}
\newcommand{\ErrorTok}[1]{\textcolor[rgb]{1.00,0.00,0.00}{\textbf{{#1}}}}
\newcommand{\NormalTok}[1]{{#1}}


\title{Plotting FLR objects with ggplot2 and ggplotFL}
\author{Iago Mosqueira, EC JRC - FLR Project}
\date{August 2013}

\begin{document}
\maketitle

\section{Using ggplot2 with FLR objects}

The \texttt{ggplot2} \footnote{\url{http://ggplot2.org/}} package
provides a powerful alternative paradigm for creating simple and complex
plots in R using the \emph{Grammar of Graphics} \footnote{Wilkinson, L.
  1999. \emph{The Grammar of Graphics}, Springer. ISBN 0-387-98774-6.}

To facilitate the use of \texttt{ggplot2} methods in \texttt{FLR}, the
\texttt{ggplotFL} package has been created. The main resources on offer
in this package are overloaded versions of the \texttt{ggplot()} method
that take directly certaing \texttt{FLR} classes, a new set of basic
plots for some \texttt{FLR} classes, based on \texttt{ggplot2} instead
of \texttt{lattice}, and some examples and documentation on how best
make use of \texttt{ggplot2}'s powerful paradigm and implementation to
obtain high quality plots for even fairly complex data structures.

\section{The overloaded \texttt{ggplot} method}

\subsection{FLQuant}

\begin{Shaded}
\begin{Highlighting}[]
\KeywordTok{ggplot}\NormalTok{(}\DataTypeTok{data =} \KeywordTok{catch}\NormalTok{(ple4), }\KeywordTok{aes}\NormalTok{(year, data)) + }\KeywordTok{geom_point}\NormalTok{() + }
    \KeywordTok{geom_line}\NormalTok{()}
\end{Highlighting}
\end{Shaded}
\begin{marginfigure}
\centering
\includegraphics{figure/flquant.pdf}
\caption{Combined line and point plot of a time series from an FLQuant
object.}
\end{marginfigure}

\subsection{FLQuants}

\begin{Shaded}
\begin{Highlighting}[]
\KeywordTok{ggplot}\NormalTok{(}\DataTypeTok{data =} \KeywordTok{FLQuants}\NormalTok{(}\DataTypeTok{Yield =} \KeywordTok{catch}\NormalTok{(ple4), }\DataTypeTok{SSB =} \KeywordTok{ssb}\NormalTok{(ple4), }
    \DataTypeTok{F =} \KeywordTok{fbar}\NormalTok{(ple4)), }\KeywordTok{aes}\NormalTok{(year, data)) + }\KeywordTok{geom_line}\NormalTok{() + }
    \KeywordTok{facet_wrap}\NormalTok{(~qname, }\DataTypeTok{scales =} \StringTok{"free"}\NormalTok{, }\DataTypeTok{nrow =} \DecValTok{3}\NormalTok{)}
\end{Highlighting}
\end{Shaded}
\begin{marginfigure}
\centering
\includegraphics{figure/flquants.pdf}
\caption{Facet wrap line plot of some time series from an FLQuants
object.}
\end{marginfigure}

\subsection{FLStock}

\section{plot() for FLR classes}

The \texttt{ggplotFL} package also provides new versions of the
\texttt{plot} method for a number of \texttt{FLR} classes. Each S4 class
defined in any \texttt{FLR} package should have a \texttt{plot()} method
defined that provides a visual summary of the contents of the object.

\subsection{FLStock}

\begin{marginfigure}
\centering
\includegraphics{figure/plotFLStock.pdf}
\caption{ggplot2 version of the standard plot() for FLStock, as applied
to \texttt{ple4}}
\end{marginfigure}

\section{Converting to data.frame}

The methods shown above simply depend on conversion of \texttt{FLR}
objects into \texttt{data.frame}, which can then be passed to
\texttt{ggplot()}. Calling \texttt{ggplot} on an \texttt{FLR} object
takes care of this conversion behind the scenes, but to obtain certains
plots, it is best to directly convert the \texttt{FLR} objects into a
\texttt{data.frame}.

\subsection{Example: plot quantiles of a simulation}

To have full control over a plot of the median (or mean) and the
confidence or probability intervals of a simulated or randomized time
series, i.e.~an \texttt{FLQuant} object with iters, we need to arrange
the different values computed from the object in separate columns of a
\texttt{data.frame}.

If we start with some random \texttt{FLQuant} object, such as

\begin{Shaded}
\begin{Highlighting}[]
\NormalTok{fla <- }\KeywordTok{rlnorm}\NormalTok{(}\DecValTok{100}\NormalTok{, }\KeywordTok{FLQuant}\NormalTok{(}\KeywordTok{exp}\NormalTok{(}\KeywordTok{cumsum}\NormalTok{(}\KeywordTok{rnorm}\NormalTok{(}\DecValTok{25}\NormalTok{, }\DecValTok{0}\NormalTok{, }
    \FloatTok{0.1}\NormalTok{)))), }\FloatTok{0.1}\NormalTok{)}
\KeywordTok{ggplot}\NormalTok{(fla, }\KeywordTok{aes}\NormalTok{(}\KeywordTok{factor}\NormalTok{(year), data)) + }\KeywordTok{geom_boxplot}\NormalTok{() + }
    \KeywordTok{xlab}\NormalTok{(}\StringTok{""}\NormalTok{)}
\end{Highlighting}
\end{Shaded}
\begin{marginfigure}
\centering
\includegraphics{figure/exsim1.pdf}
\caption{plot of chunk exsim1}
\end{marginfigure}

we can first compute the necessary statistics on the object itself, as
these operations are very efficient on an array. \texttt{quantile()} on
an \texttt{FLQuant} will return the specified quantiles along the
\texttt{iter} dimension. Let's extract the 10th, 25th, 50th, 75th and
90th quantiles.

\begin{Shaded}
\begin{Highlighting}[]
\NormalTok{flq <- }\KeywordTok{quantile}\NormalTok{(fla, }\KeywordTok{c}\NormalTok{(}\FloatTok{0.1}\NormalTok{, }\FloatTok{0.25}\NormalTok{, }\FloatTok{0.5}\NormalTok{, }\FloatTok{0.75}\NormalTok{, }\FloatTok{0.9}\NormalTok{))}
\end{Highlighting}
\end{Shaded}
The object can now be coerced to a \texttt{data.frame}

\begin{Shaded}
\begin{Highlighting}[]
\NormalTok{fdf <- }\KeywordTok{as.data.frame}\NormalTok{(flq)}
\end{Highlighting}
\end{Shaded}
and inspected to see how the 100 \texttt{iters} have been now turned
into the five requested quantiles

\begin{Shaded}
\begin{Highlighting}[]
\KeywordTok{head}\NormalTok{(fdf)}
\end{Highlighting}
\end{Shaded}
\begin{verbatim}
##   quant year   unit season   area iter  data
## 1   all    1 unique    all unique  10% 2.307
## 2   all    2 unique    all unique  10% 2.462
## 3   all    3 unique    all unique  10% 2.159
## 4   all    4 unique    all unique  10% 2.441
## 5   all    5 unique    all unique  10% 2.295
## 6   all    6 unique    all unique  10% 2.564
\end{verbatim}
The long format \texttt{data.frame} can be reshaped into a wide format
one so that we can instruct \texttt{ggplot} to use the quantiles, now in
separate columns, to provide limits for the shaded areas in
\texttt{geom\_ribbon}. To do this we can use \texttt{cast}, as follows

\begin{Shaded}
\begin{Highlighting}[]
\NormalTok{fdw <- }\KeywordTok{cast}\NormalTok{(fdf, quant + year + unit + season + area ~ }
    \NormalTok{iter, }\DataTypeTok{value =} \StringTok{"data"}\NormalTok{)}
\end{Highlighting}
\end{Shaded}
This creates a wide \texttt{data.frame} in which the \texttt{iter}
column is spread into five columns named as the levels of its conversion
into factor

\begin{Shaded}
\begin{Highlighting}[]
\KeywordTok{levels}\NormalTok{(fdf[, }\StringTok{"iter"}\NormalTok{])}
\end{Highlighting}
\end{Shaded}
\begin{verbatim}
## [1] "10%" "25%" "50%" "75%" "90%"
\end{verbatim}
We can now use those five quantile columns when plotting shaded areas
using \texttt{geom\_ribbon}. Please note that the columns names returned
by \texttt{quantile()} need to be quoted using backticks.

\begin{Shaded}
\begin{Highlighting}[]
\KeywordTok{ggplot}\NormalTok{(}\DataTypeTok{data =} \NormalTok{fdw, }\KeywordTok{aes}\NormalTok{(}\DataTypeTok{x =} \NormalTok{year, }\DataTypeTok{y =} \StringTok{`}\DataTypeTok{50%}\StringTok{`}\NormalTok{)) + }\KeywordTok{geom_line}\NormalTok{() + }
    \KeywordTok{geom_ribbon}\NormalTok{(}\KeywordTok{aes}\NormalTok{(}\DataTypeTok{x =} \NormalTok{year, }\DataTypeTok{ymin =} \StringTok{`}\DataTypeTok{10%}\StringTok{`}\NormalTok{, }\DataTypeTok{ymax =} \StringTok{`}\DataTypeTok{90%}\StringTok{`}\NormalTok{), }
        \DataTypeTok{fill =} \StringTok{"red"}\NormalTok{, }\DataTypeTok{alpha =} \FloatTok{0.15}\NormalTok{) + }\KeywordTok{geom_ribbon}\NormalTok{(}\KeywordTok{aes}\NormalTok{(}\DataTypeTok{x =} \NormalTok{year, }
    \DataTypeTok{ymin =} \StringTok{`}\DataTypeTok{25%}\StringTok{`}\NormalTok{, }\DataTypeTok{ymax =} \StringTok{`}\DataTypeTok{75%}\StringTok{`}\NormalTok{), }\DataTypeTok{fill =} \StringTok{"red"}\NormalTok{, }\DataTypeTok{alpha =} \FloatTok{0.25}\NormalTok{) + }
    \KeywordTok{ylab}\NormalTok{(}\StringTok{"data"}\NormalTok{)}
\end{Highlighting}
\end{Shaded}
\begin{marginfigure}
\centering
\includegraphics{figure/exsim7.pdf}
\caption{plot of chunk exsim7}
\end{marginfigure}

\subsection{Example: Using FLQuants}

Coercion using \texttt{as.data.frame}, combined with the use of
\texttt{cast} and \texttt{melt} (from the \texttt{reshape} package),
provides the \texttt{FLR} user with most tools required to create a
large range of \texttt{ggplot}s out of any \texttt{FLR} object.

\section{Some extra examples}

\subsection{Simulation trajectories plot}

\section{More information}

\begin{itemize}
\item
  The latest version of \texttt{ggplotFL} can always be installed using
  the \texttt{devtools} package, by calling
\end{itemize}
\begin{Shaded}
\begin{Highlighting}[]
\KeywordTok{library}\NormalTok{(devtools)}
\KeywordTok{install_github}\NormalTok{(}\StringTok{"ggplotFL"}\NormalTok{, }\StringTok{"flr"}\NormalTok{)}
\end{Highlighting}
\end{Shaded}
\begin{itemize}
\item
  To learn about ggplot2, visit the ggplot2 website \footnote{\url{http://ggplot2.org/}},
  or read the ggplot2 book.\footnote{Wickham, H. 2009. \emph{ggplot2:
    Elegant Graphics for Data Analysis}. Springer, Use R! Series. ISBN
    978-0-387-98140-6}
\end{itemize}
\section{Package versions}

\begin{itemize}
\item
  R: R version 3.0.1 (2013-05-16)
\item
  ggplot2: 0.9.3.1
\item
  ggplotFL: 2.15.20130807
\item
  FLCore: 2.5.20130820
\end{itemize}

\end{document}
